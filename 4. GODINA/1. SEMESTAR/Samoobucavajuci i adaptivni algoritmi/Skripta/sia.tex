\documentclass[12pt]{IEEEtran}
\usepackage[scr]{rsfso}
\usepackage{amsmath}
\usepackage{amsfonts}
\usepackage{pgfplots}
\usepackage{cmsrb}
\usepackage[OT2, T1]{fontenc}
\usepackage[serbian]{babel}
\usepackage{tikz}
\usepackage[justification=centering]{caption}
\usepackage{hyperref}
\usepackage{graphicx}
\usepackage{subcaption}
\usepackage{csquotes}
\usepackage{amsthm}
\usepackage{array}
\usepackage{booktabs}
\usepackage{xcolor}
\usepackage{multirow}


\title{Samoobu\v{c}avaju\'{c}i i adaptivni algoritmi}
\author{Ra\v{c}unarski upravlja\v{c}ki sistemi}

\onecolumn
\numberwithin{equation}{subsection}
\numberwithin{figure}{subsection}
\theoremstyle{definition}
\newtheorem{example}{Primjer}
\numberwithin{example}{section}
\renewcommand{\arraystretch}{1.5}
\renewcommand{\footnotesize}{\fontsize{11pt}{9pt}\selectfont}

\begin{document}

\maketitle

\tableofcontents

\newpage
\section{\textbf{U\v{c}enje potkrepljivanjem - \textit{Reinforcement learning}}}

\subsection{\textbf{Uvod u u\v{c}enje potkrepljivanjem}}

Ideja da (ljudska) bi\'{c}a u\v{c}e tako \v{s}to intereaguju sa
okolinom je prva ideja koju pomislimo kada govorimo o prirodi
u\v{c}enja. Kroz \v{z}ivot, takve interakcije su nesumnjivo
ogroman izvor na\v{s}eg znanja o okolini, kao i o nama.
\v{S}ta god radili, kakve god akcije preduzimali,
svjesni smo \v{c}injenica i reakcija okoline. \textbf{U\v{c}enje
    iz interakcija} jeste fundamentalna ideja gotovo svih teorija
inteligencije i u\v{c}enja.

\textbf{U\v{c}enje potkrepljivanjem} (\textbf{\textit{Reinforcement learning}})
jeste \textbf{tehnika u\v{c}enja} koja za cilj ima takvu \textbf{obuku agenta}
(onog koji \enquote{\textbf{u\v{c}i}}) da, u budu\'{c}nosti,
on preduzima \textbf{optimalne odluke} u zavisnosti od \textbf{stanja} u kojem se on nalazi
i \enquote{\textbf{nagrada}} (povratne informacije) iz okoline koje dobija.
Prosto re\v{c}eno, agent u\v{c}i na osnovu stanja i dobijene nagrade
\v{s}ta je najbolje uraditi.
Agentu u ovom slu\v{c}aju nije data mapa akcija koje on mora preduzeti kako bi ostvario
najve\'{c}i uspjeh - naprotiv, na njemu je
da prona\dj{}e one akcije koje maksimizuju nagradu. Na agentu je, tako\dj{}e, da prepozna
ne samo akcije koje maksimizuju trenutnu nagradu, ve\'{c} da bude svjestan toga
da trenutno preuzete akcije uti\v{c}u na sve budu\'{c}e nagrade.
Pona\v{s}anje agenta je tako odre\dj{}eno \textbf{skupom akcija} $\mathcal{A}$, a kako
ih agent preduzima, on prelazi iz jednog stanja u drugo. Stanja su opisana
\textbf{skupom stanja} $\mathcal{S}$. Agentu je tako\dj{}e, pri prelasku izme\dj{}u
datih stanja, dodjeljena nagrada iz \textbf{skupa nagrada} $\mathcal{R}$,
koja upravo govori o kvalitetu preduzete akcije.

\begin{figure}[h]
    \centering
    \begin{tikzpicture}
        \draw (3,-1) rectangle (6, 1) node[pos = 0.5] {Agent};
        \draw[->, very thick] (6, 0) -- (8, 0) -- (8, -3) -- (6, -3)
        node[xshift=85px, yshift=40px] {Akcija $a_{t}$};
        \draw (3, -4) rectangle (6, -2) node[pos = 0.5] {Okolina};
        \draw[->, thick] (3, -2.5) -- (1, -2.5) -- (1, -0.5) -- (3, -0.5)
        node[xshift=-25px, yshift=-30px] {Nagrada $r_{t}$};
        \draw[->, very thick] (3, -3.5) -- (0, -3.5) -- (0, 0.5) -- (3, 0.5)
        node[xshift=-115px, yshift=-57px] {Stanje $s_{t}$};

        \node[xshift=70px, yshift=-80px] {$r_{t + 1}$};
        \node[xshift=70px, yshift=-110px] {$s_{t + 1}$};

        \draw[dashed, thin] (2, -2) -- (2, -4) {};

    \end{tikzpicture}
    \caption{Interakcija \textbf{agent} - \textbf{okolina} u Markovljevom procesu odlu\v{c}ivanja}
\end{figure}

Dvije najbitnije karakteristike \textit{reinforcement learning}-a su \textbf{metoda poku\v{s}aja i pogre\v{s}aka}
preduzimanja akcija (neke akcije su bolje od drugih), kao i \textbf{budu\'{c}e nagrade} koje
direktno zavise od preuzetih akcija.

Za formalni zapis problema u\v{c}enja potkrepljivanjem (odnosno dinamike sistema koju posmatramo) koriste se \textbf{Markovljevi
    procesi odlu\v{c}ivanja}, te se na njih primjenjuje \textbf{najbolja akcija}
(odnosno biranje najboljih poteza - akcija) radi ostvarivanja zadatog cilja
(maksimizacije nagrade).

Jedan od izazova \textit{reinforcement learning}-a jeste to \v{s}to je potrebno postojanje
\textbf{kompromisa} izme\dj{}u \textbf{eksploatacije} i \textbf{eksploracije}.
Kako bi agent maksimizovao nagradu, on mora preduzimati
akcije koje je poku\v{s}ao u pro\v{s}losti i koje su
davale dobru nagradu, ali kako bi otkrio takve akcije,
on mora poku\v{s}ati one koje nije birao prije, odnosno
mora posjedovati eksplorativnu komponentu
kako bi verifikovao koje akcije su bolje od drugih. Dakle, agent mora
\textbf{eksploatisati} ono \v{s}to je ve\'{c} iskusio, ali
mora i \textbf{istra\v{z}ivati} kako bi pravio bolje
odluke u budu\'{c}nosti. Jedna karakteristika povla\v{c}i drugu
- agent prosto mora da proba razne akcije i da bira one koje
za njega odgovaraju.

\textit{Reinforcement learning} je dio modernog trenda vje\v{s}ta\v{c}ke
inteligencije i ma\v{s}inskog u\v{c}enja koji ima za cilj integraciju
sa teorijom optimizacije, statisti\v{c}kim ra\v{c}unom i ostalim
matemati\v{c}kim oblastima. Pored tehni\v{c}kih oblasti,
u\v{c}enje potkrepljivanjem redovno intereaguje sa psihologijom
i neuronaukama, za cilj imitiranja i razumijevanja \v{c}ovjekove sposobnosti
obzervacije i u\v{c}enja iz datih.

\subsection{\textbf{Sastavni elementi u\v{c}enja potkrepljivanjem}}

Pored \textbf{agenta} i \textbf{okoline}
u kojoj se on nalazi i sa kojom intereaguje, postoje \v{c}etiri
su\v{s}tinska dijela \textit{reinforcement learning} sistema -
\textbf{politika} (ili \textbf{na\v{c}in}) \textbf{odlu\v{c}ivanja},
\textbf{nagrada}, \textbf{vrijednost stanja} i (opciono)
\textbf{model okoline}.

\textbf{Politika odlu\v{c}ivanja} (eng. \textit{policy}) je na\v{c}in pona\v{s}anja agenta
u trenutku vremena, odnosno u odre\dj{}enom stanju. Matemati\v{c}ki
re\v{c}eno, politika odlu\v{c}ivanja je mapiranje stanja u kojem se
agent nalazi na akcije koje preduzima kada se na\dj{}e u datim
stanjima

\begin{equation}
    \pi : \mathcal{S} \to \mathcal{A}
\end{equation}

opisano re\v{c}enicom \enquote{\textit{ako sam u stanju $s_1$,
        preduzimam akciju $a_1$}} (u deterministi\v{c}kom slu\v{c}aju).

\textbf{Nagrada} defini\v{s}e cilj u kontekstu problema kojeg rje\v{s}avamo.
Pri prelasku izme\dj{}u stanja (odnosno, u svakom diskretnom vremenskom trenutku),
agent iz okoline dobija signal nagrade, koji govori o tome kako je
prethodno preduzeta akcija djelovala na okolinu, te samim tim i na
njega. Time se mo\v{z}e povu\'{c}i analogija
izme\dj{}u dobijanja \enquote{male} i \enquote{velike} nagrade sa
osje\'{c}anjem \enquote{boli} i \enquote{zadovoljstva}, u biolo\v{s}kom smislu.
Iz toga ishodi da je nagrada \textbf{mjera kvaliteta} politike odlu\v{c}ivanja,
jer iz dobijene nagrade mo\v{z}emo izvu\'{c}i zaklju\v{c}ak o tome
da li je politika koju agent sprovodi \textbf{dobra}, odnosno \textbf{optimalna}
za dati problem.

\textbf{Vrijednost stanja} jeste kumulativna suma
nagrada koje agent mo\v{z}e o\v{c}ekivati da dobije, po\v{c}ev\v{s}i
od tog stanja u kojem se trenutno nalazi. Dok nagrade odre\dj{}uju
trenutnu dobit, vrijednosti stanja predstavljaju dugoro\v{c}ni
plan dobiti kada se uzmu u obzir budu\'{c}a stanja
u kojima se agent mo\v{z}e na\'{c}i. Matemati\v{c}ki iskazano,
vrijednost stanja je suma nagrada trenutnog i svih budu\'{c}ih
stanja (mo\v{z}e ih biti beskona\v{c}no za neterminiraju\'{c}e
procese) \textbf{ponderisana te\v{z}inskim faktorom} $\gamma$

\begin{equation}
    v(s) = \begin{cases}
        \sum_{k = 0}^{T}{\gamma^{k}r_k} \text{, za deterministi\v{c}ki sistem} \\
        \mathbb{E}\left\{\sum_{k = 0}^{T}{\gamma^{k}R_k}\right\} \text{, za stohasti\v{c}ki sistem}
    \end{cases}
\end{equation}

Razlog uvo\dj{}enja pondera $\gamma$ dvojake je prirode. Naime,
prvi razlog je matemati\v{c}ki - olak\v{s}ava ra\v{c}unanje
pomenute sume koja, kao \v{s}to je re\v{c}eno,
mo\v{z}e sadr\v{z}ati beskona\v{c}no mnogo \v{c}lanova, no drugi i bitniji razlog
jeste (\textbf{ljudska}) \textbf{priroda prioritetizovanja trenutnih
nad budu\'{c}im nagradama}. Bi\'{c} su sklona davati
prioritet trenutnom stanju i trenutnim nagradama, dok ih ne
zanima \v{s}ta \'{c}e se de\v{s}avati u budu\'{c}nosti i kakve
\'{c}e nagrade tada dobijati, te se u tu svrhu
uvodi te\v{z}inski faktor $\gamma$ koji smanjuje zna\v{c}aj budu\'{c}ih
nagrada (ograni\v{c}ava se u intervalu od $0$ do $1$).
Formalna definicija svih gorenavedenih pojmova data je u nastavku teksta.

Naravno, iz prethodnog re\v{c}enog nepravilno bi bilo izvu\'{c}i
zaklju\v{c}ak da su agentu sveohuvatno bitnija
trenutna stanja i da \'{c}e njih maksimalno prioritetisati.
\v{C}itav niz dono\v{s}enja akcija zavisi od vrijednosti stanja i
agent potra\v{z}uje one akcije koje ga dovode u stanja najve\'{c}ih
vrijednosti jer to direktno pravi put akcija koje dovode do
ukupne maksimalne nagrade.

\newpage
\section{\textbf{Markovljev proces odlu\v{c}ivanja}}

\subsection{\textbf{Definicija Markovljevog procesa odlu\v{c}ivanja}}

\textbf{Markovljev proces odlučivanja} (\textit{Markov decision process}, \textbf{MDP}) je
formalizacija sekvencijalog procesa dono\v{s}enja odluka, matemati\v{c}ki
idealizovana forma \textit{reinforcement learning} problema.

Kao \v{s}to smo ve\'{c} vidjeli, onaj koji u\v{c}i i koji donosi odluke
naziva se \textbf{agent}. Stvar s kojom intereaguje i koja nije njegov
integrisani sastav jeste \textbf{okolina}. Sa pomenutom intereaguje
u sekvenci vremenskih trenutaka, $t = 0, 1, 2, ...$, donose\'{c}i akcije
koje uti\v{c}u na nju i dobijaju\'{c}i kao povratnu informaciju
novu situaciju (\textbf{stanje}) u sistemu i \textbf{nagradu},
kao reakciju na preduzetu akciju. Sekvenca kojom se ovo de\v{s}ava
formira \textbf{trajektoriju} de\v{s}avanja

\begin{gather}
    s_{0}, a_{0}, r_{0}, s_{1}, a_{1}, r_{1}, s_{2}, a_{2}, r_{2}, ...\\
    s \in \mathcal{S} \text{, } a \in \mathcal{A} \text {, } r \in \mathcal{R}
\end{gather}

Potrebno je razgrani\v{c}iti dvije vrste MDP-a:

\begin{enumerate}
    \item \textbf{Deterministi\v{c}ki Markovljev proces}

          Deterministi\v{c}ki Markovljev proces jeste model procesa u kojem su
          za svako stanje agenta ta\v{c}no odre\dj{}ene akcija koja se preduzima, kao i
          slede\'{c}e stanje, odnosno

          \begin{gather}
              s^{+} = f(s, a)\\
              r = h(s, a)
          \end{gather}

          gdje funkcije $f$ i $h$ odre\dj{}uju \textbf{deterministi\v{c}ki model}. 

          Dakle, unaprijed je mogu\'{c}e odrediti trajektoriju sistema bazirano na
          trenutnom stanju agenta.

    \item \textbf{Stohasti\v{c}ki Markovljev proces}

        Stohasti\v{c}ki Markovljev proces je uop\v{s}tenje deterministi\v{c}kog.
        Naime, u zavisnosti od trenutnog stanja i akcije koju agent preduzme,
        postoji vjerovatno\'{c}a da \'{c}e on zavr\v{s}iti u nekom od narednih
        stanja i dobiti neku nagradu, odnosno

        \begin{equation}
            p(s^{+}, r | s, a) = \mathbb{P}\{S^{+} = s^{+}, R = r | S = s, A = a\}
        \end{equation}

        gdje se vjerovatno\'{c}a $p$ naziva \textbf{stohasti\v{c}kim modelom}, 
        \textbf{modelom prelaza}, \textbf{vjerovatno\'{c}a prelaza},
        \textbf{dinamika sistema}.

        Dakle, nije mogu\'{c}e odrediti ta\v{c}nu trajektoriju sistema baziranu
        na trenutnom stanju, ve\'{c} je ona u zavisnosti od raspodjele vjerovatno\'{c}a.

\end{enumerate}

\subsection{\textbf{Dobitak} (\textit{Gain}), \textbf{politika odlu\v{c}ivanja, vrijednost stanja, vrijednost akcije}}

U procesu interakcije sa okolinom, agent kao povratni signal dobija nagradu.
\textbf{Ukupna dobijena nagrada} (\textit{gain}) ra\v{c}una se kao suma sekvence svih
dobijenih nagrada po\v{c}ev\v{s}i od trenutnog momenta i
nastavljaju\'{c}i se u budu\'{c}nost, odnosno

\begin{equation}
    g = \begin{cases}
        \sum_{k = 0}^{T}{\gamma^{k}r_{k}} \text{, za deterministi\v{c}ki sistem} \\
        \mathbb{E}\left\{\sum_{k = 0}^{T}{\gamma^{k}R_{k}}\right\} \text{, za stohasti\v{c}ki sistem}
    \end{cases}
\end{equation}

Agent intereaguje sa okolinom donose\'{c}i odluke. Pravilo
pod kojim se donose odluke naziva se \textbf{politika odlu\v{c}ivanja}

\begin{equation}
    a = \begin{cases}
        \pi(s) \text{, za deterministi\v{c}ke politike} \\
        \pi(a | s) = \mathbb{P}\{A = a | S = s\} \text{, za stohasti\v{c}ke politike}
    \end{cases}
\end{equation}

i ona o\v{c}igledno zavisi od stanja u kojem se agent nalazi.
Tako\dj{}e, valjalo bi primjetiti da je mogu\'{c}e primjeniti ili deterministi\v{c}ku ili
stohasti\v{c}ku politiku, nezavisno od toga da li je sam proces
deterministi\v{c}ki ili stohasti\v{c}ki - \textbf{politika odlu\v{c}ivanja
ne zavisi od karakteristika modela}.

Ako fiksiramo po\v{c}etno stanje, mo\v{z}emo pri\v{c}ati o
\textbf{o\v{c}ekivanoj nagradi} u zavisnosti od politike
odlu\v{c}ivanja $\pi$

\begin{equation}
    g_{\pi}(s_{0}) = \mathbb{E}_{\pi}\left\{\sum_{k = 0}^{T}{\gamma^{k}R_{k}} | S_{0} = s_{0}\right\}
\end{equation}

Tada mo\v{z}emo definisati \textbf{vrijednost stanja}

\begin{equation}
    v_{\pi}(s) = g_{\pi}(s)
\end{equation}

koje odre\dj{}uje \textbf{o\v{c}ekivanu nagradu po\v{c}ev\v{s}i od
    stanja $s$}.

Tako\dj{}e, defini\v{s}e se i \textbf{vrijednost akcije}

\begin{equation}
    q_{\pi}(s, a) = g_{\pi}(s | s_{0} = s, a_{0} = a) = r_{0} + \gamma g_{\pi}(s^{+}) = r_{0} + \gamma q_{\pi}(s^{+}, a^{+})
\end{equation}

koja govori o \textbf{nagradi} (ozna\v{c}ena
sa $r_{0}$) \textbf{dobijenoj kada se krene iz po\v{c}etnog
stanja $s_{0} = s$ i preduzme akcija $a_{0} = a$}, te se \textbf{nakon toga slijedi politika $\pi$}.
Dakle, po\v{c}etna akcija $a$ nezavisna je od politike $\pi$,
no mo\v{z}e biti njen dio (bez gubitka op\v{s}tosti), te zato imamo puno pravo pisati 
$g_{\pi}(s^{+}) = q_{\pi}(s^{+}, a^{+})$.

\begin{example}
    \textbf{Odre\dj{}ivanje vrijednosti stanja}

    Posmatrajmo dati sistem odlu\v{c}ivanja

    \begin{figure}[h]
        \centering
        \begin{tikzpicture}
            \draw[blue!60, fill=blue!20] (-2, -0.5) circle (1);
            \draw[red!60, fill=red!20] (2, -4) circle (1);
            \draw[red!60, fill=red!20] (3, -1) circle (1);
            \draw[red!60, fill=red!20] (2, 2) circle (1);

            \node at (-3.5, -0.5) {$s^{0}$};
            \node at (3.5, -4) {$s^{1}$};
            \node at (4.5, -1) {$s^{2}$};
            \node at (3.5, 2) {$s^{3}$};

            \node at (2, -4) {$10$};
            \node at (3, -1) {$20$};
            \node at (2, 2) {$15$};

            % \draw[->, very thick] (-1.134, 0) to[out=30] (1, 2);
            \path[->, very thick, out=-3] (-1, -0.5) edge (1, 2);
            \path[->, very thick, out=0.5] (-1, -0.5) edge (2, -1);
            \path[->, very thick, out=2] (-1, -0.5) edge (1, -4);

            \node at(-0.5, 1) {$a^{3}$};
            \node at(0.5, 0) {$a^{2}$};
            \node at(-0.5, -2) {$a^{1}$};

            \node at(0, 2.5) {$r^{3} = 5$};
            \node at(1, -1) {$r^{2} = -5$};
            \node at(0, -4) {$r^{1} = 2$};

        \end{tikzpicture}
        \caption{Primjer 1.}
    \end{figure}

    \begin{table}[h]
        \normalsize
        \centering
        \begin{tabular}{*4c}
            \textbf{Akcija $a$} & \textbf{Stanje $s^{+}$} & \textbf{Vrijednost stanja $v(s^{+})$} & \textbf{Nagrada $r$} \\
            \toprule
            $a^{1}$             & $s^{1}$                 & $10$                                  & $2$                  \\
            \hline
            $a^{2}$             & $s^{2}$                 & $20$                                  & $-5$                 \\
            \hline
            $a^{3}$             & $s^{3}$                 & $15$                                  & $5$                  \\
            \hline
        \end{tabular}
    \end{table}

    Dakle, na raspolaganju su nam date tri akcije, $\mathcal{A} = \left\{a^{1}, a^{2}, a^{3}\right\}$. Nagrada koju mo\v{z}emo
    o\v{c}ekivati da dobijemo po\v{c}ev\v{s}i u stanju $s^{0}$ jeste, po formuli

    \begin{equation}
        g(s) = \sum_{k = 0}^{T}{\gamma^{k}r_{k}} = r_{0} + \gamma r_{1} + \gamma^{2} r_{2} + \gamma^{3} r_{3} ... = r_{0} + \gamma (r_{1} + \gamma r_{2} + \gamma^{2} r_{3}...)
    \end{equation}

    Primjetimo da je red $r_1 + \gamma r_2 + ...$ zapravo \textbf{dobijena nagrada}\footnote{
        Subskript ozna\v{c}ava vremenske trenutke, te je za stanje $s^{+}$ zapravo po\v{c}etni 
        trenutak $t = 1$ i formula za dobijenu nagradu ostaje validna.}
    za stanje $s^{+}$, te \'{c}e va\v{z}iti da je

    \begin{equation}
        g(s) = r_{0} + \gamma g(s^{+})
    \end{equation}

    Dobijeni rezultat predstavlja \textbf{rekurzivnu formulu za ra\v{c}unanje dobijene nagrade}, a kako je, za konkretnu
    politiku odlu\v{c}ivanja $v_{\pi}(s) = g_{\pi}(s)$, slijedi da je

    \begin{equation}
        v_{\pi}(s) = r_{0} + \gamma v_{\pi}(s^{+})
    \end{equation}

    U zavisnosti od preduzete akcije mo\v{z}emo izra\v{c}unati
    vrijednost stanja $s^{0}$ (za parametar $\gamma$ uzeta je
    vrijednost $0.9$)

    \begin{table}[h]
        \normalsize
        \centering
        \begin{tabular}{*3c}
            \textbf{Akcija $a$} & \textbf{Stanje $s^{+}$} & \textbf{Vrijednost stanja $v(s^{0})$} \\
            \toprule
            $a^{1}$             & $s^{1}$                 & $11$                                  \\
            \hline
            $a^{2}$             & $s^{2}$                 & $13$                                  \\
            \hline
            $a^{3}$             & $s^{3}$                 & $18.5$                                \\
            \hline
        \end{tabular}
    \end{table}

\end{example}

Prethodni primjer dobar je pokazatelj preduzimanja \textbf{optimalne politike odlu\v{c}ivanja}, koja se defini\v{s}e 
kao \textbf{politika koja \'{c}e donijeti najve\'{c}u nagradu} od svih mogu\'{c}ih politika $\pi \in \mathcal{P}$
\begin{gather}
    v^{*}_{\pi}(s) = \max_{\pi \in \mathcal{P}}\{v_{\pi}(s)\} = \max_{\pi \in \mathcal{P}}\{r_{0} + \gamma v^{*}_{\pi}(s^{+})\} = \max_{\pi \in \mathcal{P}}\{h(s, a) + \gamma v^{*}_{\pi}(f(s, a))\}\\
    a = \pi(s)
\end{gather}

Konkretno za primjer, optimalna politika bi bila akcija $a^{3}$, koja bi
vrijednost stanja $s^{0}$ postavila na $18.5$.

Sli\v{c}an rezultat dobijamo ra\v{c}unaju\'{c}i vrijednosti akcija

\begin{table}[h]
    \normalsize
    \centering
    \begin{tabular}{*2c}
        \textbf{Akcija $a$} & \textbf{Vrijednost akcije $q(s^{0}, a$)} \\
        \toprule
        $a^{1}$             & $11$                                     \\
        \hline
        $a^{2}$             & $13$                                     \\
        \hline
        $a^{3}$             & $18.5$                                   \\
        \hline
    \end{tabular}
\end{table}

te bismo \textbf{optimalnu politiku koja donosi najve\'{c}u vrijednost akcije} ra\v{c}unali kao

\begin{gather}
    q^{*}_{\pi}(s, a) = \max_{\pi \in \mathcal{P}}\{q_{\pi}(s, a)\} = \max_{\pi \in \mathcal{P}}\{r_{0} + q^{*}_{\pi}(s^{+}, a^{+})\} = \max_{\pi \in \mathcal{P}}\{h(s, a) + q^{*}_{\pi}(f(s, a), a^{+})\}\\
    a^{+} = \pi(s^{+})
\end{gather}

\newpage
\section{\textbf{Na\v{c}ini ra\v{c}unanja vrijednosti stanja i vrijednosti akcija. Optimalna politika odlu\v{c}ivanja}}

\subsection{\textbf{Iterativno ra\v{c}unanje vrijednosti stanja i vrijednosti akcija}}

\begin{example}
    \textbf{Evaluacija vrijednosti stanja na osnovu konkretne politike}

    Pogledajmo slede\'{c}i primjer\footnote{
        Jarko crvenom bojom ozna\v{c}eno je \textbf{terminalno} \enquote{\textbf{stanje}}. Ono zapravo
        nije su\v{s}tinski stanje, jer u njemu ne mo\v{z}emo donositi akcije -
        ozna\v{c}ava kraj \enquote{igre}. Obi\v{c}no uzimamo njegovu vrijednost za $0$.
    }

    \begin{figure}[h]
        \centering
        \begin{tikzpicture}
            \draw[red!60, fill=red!20] (-2, 2) circle (1);
            \draw[red!60, fill=red!20] (3, 3) circle (1);
            \draw[red!60, fill=red!20] (2, -2) circle (1);
            \draw[red!60, fill=red!20] (6, 0) circle (1);
            \draw[red!100, fill=red!60, ultra thick] (8, 3.5) circle (1);

            \node at(-2, 2) {$s^{0}$};
            \node at(3, 3) {$s^{3}$};
            \node at(2, -2) {$s^{1}$};
            \node at(6, 0) {$s^{2}$};

            \path[blue, ->, very thick, out=2, dashed] (-1, 2) edge (2, 3);
            \path[orange, ->, very thick, out=2] (-1, 2) edge (1, -2);
            \path[orange, ->, very thick, in=-50] (2, -3) edge[bend left=120, looseness=3] (1, -2);
            \draw[blue, ->, very thick, dashed] plot[smooth, tension=2] coordinates {(3, -2) (4, -1.3) (5, -1.5) (6, -1)};
            \path[orange, ->, very thick] (6, -1) edge[bend left=90, bend angle=20] (3, -2);
            \path[orange, ->, very thick, out=2] (4, 3) edge (5, 0);
            \path[blue, ->, very thick, out=2, dashed] (4, 3) edge (7, 3.5);
            \path[blue, ->, very thick, in=0.5, dashed] (7, 0) edge (8, 2.5);

            \node at(0.5, 3.5) {$-1$};
            \node at(-0.5, 0) {$-2$};
            \node at(1.25, -4) {$-1$};
            \node at(4.5, -3) {$-2$};
            \node at(4.25, -0.75) {$-1$};
            \node at(3.75, 1.5) {$-1$};
            \node at(5.5, 4.25) {$-3$};
            \node at(7.25, 1.25) {$-1$};
        \end{tikzpicture}
        \caption{Primjer 1.}
    \end{figure}

    Neka je politika $\pi$

    \begin{equation}
        \pi(s) = "\begingroup\color{blue}\mathbf{\rightarrow}\endgroup"
    \end{equation}

    odnosno \enquote{\textbf{biraj plavu} (\textbf{isprekidanu}) \textbf{akciju}} .

    Krenuv\v{s}i unazad, odnosno od terminalnog stanja,
    vrijednosti svakog od stanja su

    \begin{gather}
        v^{3} = -3 + \gamma v^{\text{term}}\\
        v^{2} = -1 + \gamma v^{\text{term}}\\
        v^{1} = -1 + \gamma v^{2}\\
        v^{0} = -1 + \gamma v^{3}
    \end{gather}

    O\v{c}igledno je da ako izaberemo $v^{\text{term}} = 0$,
    dobijamo vrijednosti za $v^{2}$ i $v^{3}$, uz pomo\'{c}
    kojih mo\v{z}emo izra\v{c}unati vrijednosti preostalih stanja
    $v^{0}$ i $v^{1}$. \textbf{Algoritam evaluacije vrijednosti stanja} je tako

    \begin{enumerate}
        \item \textbf{Izabrati vrijednost $v^{\text{term}}$}
        \item \textbf{Prona\'{c}i i evaluirati sva stanja iz kojih je mogu\'{c}e do\'{c}i u terminalno}
        \item \textbf{Prona\'{c}i i evaluirati sva stanja iz kojih je mogu\'{c}e do stanja ve\'{c} poznatih vrijednosti}
        \item \textbf{Ponoviti korak 3.}
    \end{enumerate}

    
    Kada imamo ogroman skup stanja i slo\v{z}eniju politiku odlu\v{c}ivanja,
    potrebno je primjeniti numeri\v{c}ke iterativne metode
    za odre\dj{}ivanje vrijednosti stanja. Ako matri\v{c}no zapi\v{s}emo
    prethodne jedna\v{c}ine

    \begin{equation}
        \begin{bmatrix}
            v^{0} \\
            v^{1} \\
            v^{2} \\
            v^{3}
        \end{bmatrix} =
        \gamma
        \begin{bmatrix}
            0 & 0 & 0 & 1 \\
            0 & 0 & 1 & 0 \\
            0 & 0 & 0 & 0 \\
            0 & 0 & 0 & 0
        \end{bmatrix}
        \begin{bmatrix}
            v^{0} \\
            v^{1} \\
            v^{2} \\
            v^{3}
        \end{bmatrix}
        + \gamma
        \begin{bmatrix}
            0 \\
            0 \\
            1 \\
            1
        \end{bmatrix}
        v^{\text{term}}
        +
        \begin{bmatrix}
            -1 \\
            -1 \\
            -1 \\
            -3
        \end{bmatrix}
    \end{equation}

    Direktno rje\v{s}avanje po matrici $\mathbf{v}$ dobija se formulom

    \begin{equation}
        \mathbf{v} = (\mathbf{I} - \gamma \mathbf{A})^{-1}\mathbf{r}
    \end{equation}

    ili iterativnom metodom

    \begin{equation}
        \mathbf{v}^{k + 1} = \gamma \mathbf{Av}^{k} + \mathbf{r}
    \end{equation}

    gdje je $\mathbf{v}^{0}$ nasumi\v{c}no inicijalizovano.

\end{example}


\subsection{\textbf{Deterministi\v{c}ke i stohasti\v{c}ke jedna\v{c}ine vrijednosti stanja i akcija}}

Podsjetimo se rekurzivnih formula za ra\v{c}unanje vrijednosti stanja
i vrijednost akcija u deterministi\v{c}kom slu\v{c}aju

\begin{gather}
    v_{\pi}(s) = h(s, \pi(s)) + \gamma v_{\pi}(f(s, \pi(s)))\\
    q_{\pi}(s, a) = h(s, a) + \gamma q_{\pi}(f(s, a), \pi(f(s, a)))
\end{gather}

U stohasti\v{c}kom slu\v{c}aju govorimo o \textbf{o\v{c}ekivanoj nagradi},
odnosno

\begin{equation}
    g_{\pi}(s) = \mathbb{E}\left\{\sum_{k = 0}^{T}{\gamma^{k}R_{k}} | S_{0} = s\right\}
\end{equation}

Razlo\v{z}imo izraz na slede\'{c}i na\v{c}in

\begin{equation}
    \begin{gathered}
        g_{\pi}(s) = \mathbb{E_{\pi}}\left\{\sum_{k = 0}^{T}{\gamma^{k}R_{k}} | S_{0} = s\right\} = \mathbb{E_{\pi}}\left\{R_{0} + \gamma \sum_{k = 1}^{T}{\gamma^{k - 1}R_{k}} | S_{0} = s\right\} = \\
        \sum_{a \in \mathcal{A}}{\pi(a | s)\mathbb{E_{\pi}}\left\{R_{0} + \gamma \sum_{k = 1}^{T}{\gamma^{k - 1}R_{k}} | S_{0} = s, A_{0} = a\right\}} =\\
        \sum_{a \in \mathcal{A}}{\pi(a | s) \sum_{\substack{s^{+} \in \mathcal{S}\\r \in \mathcal{R}}}{p(s^{+}, r | s, a) \mathbb{E_{\pi}}\left\{R_{0} + \gamma \sum_{k = 1}^{T}{\gamma^{k - 1}R_{k}} | S_{0} = s, A_{0} = a, S_{1} = s^{+}, R_{0} = r\right\}}} =\\
        \sum_{a \in \mathcal{A}}{\pi(a | s) \sum_{\substack{s^{+} \in \mathcal{S}\\r \in \mathcal{R}}}{p(s^{+}, r | s, a) \left[r + \gamma \mathbb{E_{\pi}} \left\{ \sum_{k = 1}^{T}{\gamma^{k - 1}R_{k}} | S_{1} = s^{+} \right\} \right]}}
    \end{gathered}
\end{equation}

Poslednja formula mo\v{z}e se zapisati kao

\begin{equation}
    v_{\pi}(s) = \sum_{a \in \mathcal{A}}{\pi(a | s) \sum_{\substack{s^{+} \in \mathcal{S}\\r \in \mathcal{R}}}{p(s^{+}, r | s, a) \left[r + \gamma v_{\pi}(s^{+}) \right]}}
\end{equation}

i predstavlja \textbf{jedna\v{c}inu vrijednosti stanja u stohasti\v{c}kom slu\v{c}aju}.

\v{S}to se ti\v{c}e vrijednosti akcija

\begin{equation}
    q_{\pi}(s, a) = \mathbb{E_{\pi}}\left\{R_{0} + \gamma \sum_{k = 1}^{T}{\gamma^{k - 1}R_{k}} | S_{0} = s, A_{0} = a\right\}
\end{equation}

Ako raspi\v{s}emo dalje

\begin{equation}
    \begin{gathered}
        q_{\pi}(s, a) = \mathbb{E_{\pi}}\left\{R_{0} + \gamma \sum_{k = 1}^{T}{\gamma^{k - 1}R_{k}} | S_{0} = s, A_{0} = a\right\} =\\
        \sum_{\substack{s^{+} \in \mathcal{S}\\r \in \mathcal{R}}}{p(s^{+}, r | s, a) \mathbb{E_{\pi}}\left\{ R_{0} + \gamma \sum_{k = 1}^{T}{\gamma^{k - 1} R_{k}} | S_{0} = s, A_{0} = a, S_{1} = s^{+}, R_{0} = r \right\}} =\\
        \sum_{\substack{s^{+} \in \mathcal{S}\\r \in \mathcal{R}}}{p(s^{+}, r | s, a) \left[ r + \gamma \mathbb{E_{\pi}}\left\{ \sum_{i = 1}^{T}{\gamma^{k - 1}R_{k}} | S_{1} = s^{+} \right\} \right]} =\\
        \sum_{\substack{s^{+} \in \mathcal{S}\\r \in \mathcal{R}}}{p(s^{+}, r | s, a) \left[ r + \gamma \sum_{a^{+} \in \mathcal{A}}{\pi(a^{+} | s^{+}) \mathbb{E_{\pi}}\left\{ \sum_{k = 1}^{T}{\gamma^{k - 1}R_{k}} | S_{1} = s^{+}, A_{1} = a^{+} \right\}}\right]}
    \end{gathered}
\end{equation}

Poslednja formula mo\v{z}e se zapisati kao

\begin{equation}
    q_{\pi}(s, a) = \sum_{\substack{s^{+} \in \mathcal{S}\\r \in \mathcal{R}}}{p(s^{+}, r | s, a) \left[ r + \gamma \sum_{a^{+} \in \mathcal{A}}{\pi(a^{+} | s^{+})q_{\pi}(s^{+}, a^{+})} \right] }
\end{equation}

i predstavlja \textbf{jedna\v{c}inu vrijednosti stanja u stohasti\v{c}kom slu\v{c}aju}.

Prethodne jedna\v{c}ine nazivaju se \textbf{Belmanovim jedna\v{c}inama}.

\newpage
\begin{example}
\textbf{Primjer odre\dj{}ivanja vrijednosti stanja i akcija na stohasti\v{c}kom MDP-u}

Razmotrimo slede\'{c}i primjer 

\begin{figure}[h]
    \centering
    \begin{tikzpicture}
        \draw[blue!60, fill=blue!20] (-2, -0.5) circle (1);
        \draw[red!60, fill=red!20] (2, -4) circle (1);
        \draw[red!60, fill=red!20] (3, -1) circle (1);
        \draw[red!60, fill=red!20] (2, 2) circle (1);

        \node at (-2, -0.5) {$s^{0}$};
        \node at (2, -4) {$s^{3}$};
        \node at (3, -1) {$s^{2}$};
        \node at (2, 2) {$s^{1}$};

        \path[blue, ->, very thick, out=-3, dashed] (-1, -0.5) edge (1, 2);
        \path[blue, ->, very thick, in=-50, dashed] (-2, -1.5) edge[bend left=120, looseness=3] (-3, -0.5);
        \path[red, ->, very thick, out=0.5] (-1, -0.5) edge (2, -1);
        \path[red, ->, very thick, out=2] (-1, -0.5) edge (1, -4);

        \node at(-2.5, -2.5) {$0.9$};
        \node at(-0.5, 1) {$0.1$};
        \node at(0.75, 0) {$0.75$};
        \node at(-0.5, -2) {$0.25$};

        \node[draw, dotted, very thick] at(-4.25, -1) {$-1$};
        \node[draw, dotted, very thick] at(-0.25, 2.25) {$-2$};
        \node[draw, dotted, very thick] at(1.25, -1.25) {$-2$};
        \node[draw, dotted, very thick] at(0, -4) {$-3$};

        \node at(-4, 0) {$v^{1} = -1$};
        \node at(4, -4) {$v^{3} = -3$};
        \node at(5, -1) {$v^{2} = -2$};
        \node at(4, 2) {$v^{1} = -1$};

    \end{tikzpicture}
    \caption{Primjer 2.}
\end{figure}

Decimalne vrijednosti na slici iznad ozna\v{c}avaju vjerovatno\'{c}u
zavr\v{s}etka u odre\dj{}enim stanjima ako se
biraju konkretne akcije, dok su uokvirene vrijednosti 
nagrade koje agent dobija igranjem odre\dj{}enih akcija. Tako\dj{}e, 
poznate su nam vrijednosti stanja $s^{1}, s^{2}$ i $s^{3}$.

Neka je politika odlu\v{c}ivanja \enquote{\textbf{Ako si u stanju 
$s^{0}$, sa $30\%$ biraj akciju 
\enquote{$\begingroup \color{blue} \mathbf{\rightarrow} \endgroup$}} (\textbf{plavu isprekidanu akciju}),
\textbf{dok sa $70\%$ biraj akciju \enquote{$\begingroup \color{red} \mathbf{\rightarrow} \endgroup$}} 
(\textbf{crvenu punu akciju})}. Matemati\v{c}ki zapisano 

\begin{equation}
    \pi(a | s) = \begin{cases}
        0.3 \text{, } s = s^{0} \rightarrow a = \text{\textit{Isprekidana plava}}\\
        0.7 \text{, } s = s^{0} \rightarrow a = \text{\textit{Puna crvena}}
    \end{cases} 
\end{equation}

Ako zapi\v{s}emo tabli\v{c}no \v{s}ta vidimo na slici iznad

\begin{table}[h]
    \normalsize
    \centering
    \begin{tabular}{*4c}
        \textbf{Stanje} & \textbf{Akcija} & \textbf{Prelaz u stanje} & $p(s^{+}, r | s, a)$\\      
        \toprule 
        \multirow{4}{*}{$s^{0}$} & \textit{Isprekidana plava} & $s^{0}$ & $p(s^{+} = s^{0}, r = -1 | s = s^{0}, a = \text{\textit{Ispr. plava}}) = 0.9$\\
        \cline{2-4}
        & \textit{Isprekidana plava} & $s^{1}$ & $p(s^{+} = s^{1}, r = -2 | s = s^{0}, a = \text{\textit{Ispr. plava}}) = 0.1$\\  
        \cline{2-4}
        & \textit{Puna crvena} & $s^{2}$ & $p(s^{+} = s^{2}, r = -2 | s = s^{0}, a = \text{\textit{Puna crv.}}) = 0.75$\\  
        \cline{2-4}
        & \textit{Puna crvena} & $s^{3}$ & $p(s^{+} = s^{3}, r = -3 | s = s^{0}, a = \text{\textit{Puna crv.}}) = 0.25$\\  
        \cline{2-4}
    \end{tabular}
    
\end{table}

Dakle, imamo \textbf{stohasti\v{c}ku politiku odlu\v{c}ivanja}, kao 
i \textbf{stohasti\v{c}ku okolinu}, te \'{c}emo primjeniti formulu 

\begin{equation}
    v_{\pi}(s) = \sum_{a \in \mathcal{A}}{\pi(a | s) \sum_{\substack{s^{+} \in \mathcal{S}\\r \in \mathcal{R}}}{p(s^{+}, r | s, a) \left[r + \gamma v_{\pi}(s^{+}) \right]}}
\end{equation}

Prvo rje\v{s}imo unutra\v{s}nju sumu. To radimo fiksiranjem po akcijama, odnosno posmatraju\'{c}i \v{s}ta se
de\v{s}ava kada izaberemo partikularnu akciju $a$

\begin{gather}
    v^{0}_{\text{Ispr. pl.}} = 0.9(-1 + \gamma v^{0}_{\text{Ispr. pl.}}) + 0.1(-2 + \gamma v^{1})\\
    v^{0}_{\text{Puna crv.}} = 0.75(-2 + \gamma v^{2}) + 0.25(-3 + \gamma v^{3})
\end{gather}

te sabiranjem dobijenih rezultata ponderisanih vjerovatno\'{c}ama $\pi$ 

\begin{equation}
    v^{0} = 0.3v^{0}_{\text{Ispr. pl.}} + 0.7v^{0}_{\text{Puna crv.}} 
\end{equation}

\end{example}

\subsection{\textbf{Belmanove jedna\v{c}ine optimalne politike}}

Zapisano opet, jedna\v{c}ine optimalnih politika u zavisnosti od vrijednosti
stanja i vrijednosti akcija

\begin{gather}
    v^{*}_{\pi}(s) = \max_{\pi \in \mathcal{P}}\{h(s, a) + \gamma v^{*}_{\pi}(f(s, a))\} \text{, gdje } a = \pi(s)\\
    q^{*}_{\pi}(s, a) = \max_{\pi \in \mathcal{P}}\{h(s, a) + \gamma q^{*}_{\pi}(f(s, a), a^{+})\} \text{, gdje } a^{+} = \pi(s^{+}) = \pi(f(s, a))
\end{gather}

Kako su politike konstruisane od akcija, mi su\v{s}tinski za svako stanje
moramo prona\'{c}i onakvu akciju \v{c}ija primjena dovodi do
najve\'{c}e nagrade, te se \textbf{pronala\v{z}enja optimalnih politika odlu\v{c}ivanja}
svodi na \textbf{pronala\v{z}enje optimalnih akcija}.

Dakle, to zna\v{c}i da se prethodno napisana formula za evaluaciju vrijednosti stanja mo\v{z}e  zapisati kao

\begin{equation}
    v^{*}(s) = \max_{a \in \mathcal{A}}\left\{ h(s, a) + \gamma v^{*}(f(s, a)) \right\}
\end{equation}

\v{S}to se ti\v{c}e optimalne vrijednosti akcija,
optimalna politika se slijedi tek nakon prvog koraka
(prva akcija nije zavisna od politike odlu\v{c}ivanja),
te se jedna\v{c}ina optimuma vrijednosti akcija mo\v{z}e
zapisati kao

\begin{equation}
    q^{*}(s, a) = h(s, a) + \gamma\max_{a^{+} \in \mathcal{A}}\left\{q^{*}(f(s, a), a^{+})\right\}
\end{equation}

Prethodne jedna\v{c}ine predstavljaju \textbf{Belmanove jedna\v{c}ine optimalnosti}.

Stohasti\v{c}ke verzije jedna\v{c}ina optimalnosti su 

\begin{gather}
    v^{*}(s) = \max_{a \in \mathcal{A}}\left\{ \sum_{\substack{s^{+} \in \mathcal{S}\\r \in \mathcal{R}}}{p(s^{+}, r | s, a) \left[ r + \gamma v^{*}(s^{+}) \right]}\right\}\\
    q^{*}(s, a) = \sum_{\substack{s^{+} \in \mathcal{S}\\r \in \mathcal{R}}}{p(s^{+}, r | s, a)\left[ r + \gamma \max_{a^{+} \in \mathcal{A}} \left\{ q^{*}(s^{+}, a^{+}) \right\} \right]}
\end{gather}

\end{document}